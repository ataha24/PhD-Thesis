\onehalfspacing
Stereotactic neurosurgery is concerned with the precise targeting of brain structures, where millimetric deviations can be the difference between optimal therapy and adverse outcomes. Deep brain stimulation (DBS) is a stereotactic procedure that involves electrode implantation into subcortical brain regions to modulate dysfunctional neural circuits. Enabled by advances in neuroimaging and surgical technology, DBS has been applied across a broad range of neurological and psychiatric disorders, with over 250,000 cases performed worldwide. However, millimetric localization of small subcortical regions remains difficult. Clinical MRI is often limited by resolution, contrast, and susceptibility artifacts, while nonlinear registration—considered the gold standard for automatic localization—yields spatial errors on the order of 1–5 millimeters. These errors are clinically meaningful given the anatomical complexity of DBS targets. As a result, there is growing demand for tools that support millimetric localization across imaging modalities and populations. Developments in machine learning (ML) offer opportunities to address challenges in brain structure localization by improving accuracy, automation, and scalability but require further validation to meet the spatial demands of neurosurgical targeting. This dissertation leverages brain coordinates inspired by classical stereotactic principles and ML to develop an open software infrastructure that supports spatially precise and interpretable workflows in neuroimaging and neurosurgery. In \textbf{Chapter 2}, we validate a protocol for brain coordinate annotation and openly release 6,000 coordinates placed on structural MRI across multiple field strengths and clinical populations. We demonstrate the utility of this protocol and data for millimetric quality control of neuroimaging workflows. In \textbf{Chapter 3}, we use this data to develop AutoAFIDs, an open-source application for automatic landmark detection that achieves millimetric localization accuracy. We showcase the broader utility of AutoAFIDs by building modular applications for registration quality control and morphometric brain charting. In \textbf{Chapter 4}, we extend this coordinate-based framework to predict the location of subcortical brain regions, demonstrating millimetric accuracy on clinical MRI and validate predictions using ultra-high field MRI where ground truth anatomy is visible. Together, the tools and datasets developed in this dissertation establish an open and generalizable infrastructure for surgical targeting, image validation, and population-level brain mapping.
