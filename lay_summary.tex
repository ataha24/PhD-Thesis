\onehalfspacing
Imagine walking into a dark room and trying to plug in your phone charger. Without light, you can depend on known landmarks—a chair, a wall—to guide you. In neurosurgery, a similar problem arises during deep brain stimulation (DBS), a treatment that acts like a pacemaker for the brain. DBS delivers electrical impulses through millimeter-wide electrodes implanted in deep brain regions to restore disrupted brain activity in conditions like Parkinson’s disease. But the targets for DBS are sometimes small and hard to see on clinical MRI scans. A deviation of just 1–2 millimeters can reduce effectiveness or cause side effects. To guide electrode placement, surgeons can rely on visible brain landmarks—structures that serve as references in the dark. This dissertation combines neuroanatomical landmarks with artificial intelligence (AI) to support more accurate and interpretable surgical planning. In \textbf{Chapter 2}, we developed and validated a protocol for placing 32 anatomical brain landmarks—called anatomical fiducials—on MRI scans. These fiducials were annotated across a wide range of patients, scanners, and clinical sites, resulting in a public dataset of over 6,000 points. This framework enables millimetric quality control of neuroimaging workflows, helping identify millimetric errors introduced during image processing. In \textbf{Chapter 3}, we developed \texttt{AutoAFIDs}, an open-source AI tool that detects these landmarks automatically and with high precision. We used this tool to chart how brain shape varies across the lifespan, revealing subtle changes with age, sex, and disease. This is analogous to studying the blueprints in our dark room analogy which ultimately can help surgeons navigable brain anatomy more reliably. In \textbf{Chapter 4}, we extended the fiducial landmark framework to estimate the location of DBS targets. By learning the spatial relationship between visible landmarks and hidden targets, we trained models that could predict target locations on clinical MRI. To validate our findings, we made use of ultra-high resolution 7-Tesla MRI—which is analogous to using a flashlight in our dark room analogy. Together, this work delivers open and generalizable tools to support surgical targeting, brain mapping, and neuroimaging validation—helping neurosurgeons navigate the brain with greater precision and confidence.

