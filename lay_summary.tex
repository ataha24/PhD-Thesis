\onehalfspacing
Imagine walking into a dark room and trying to plug in your phone charger. Without light, you rely on familiar landmarks—a chair, a wall—to guide you. In neurosurgery, a similar challenge arises during deep brain stimulation (DBS), a treatment that modulates abnormal brain circuits using millimeter-wide electrodes—functioning like a pacemaker for the brain. DBS targets are small and often hard to visualize on clinical imaging. Millimeter deviations in electrode placement can compromise treatment or cause side effects. To guide electrode placement, surgeons can depend on visible anatomical landmarks—reference points that help navigate the brain, much like landmarks in a dark room. This dissertation combines brain landmarks with artificial intelligence (AI) to support accurate neurosurgical planning and interpretable brain mapping. In \textbf{Chapter} \textbf{\ref{chap:afidsdata}}, we curated and validated a protocol describing the location of 32 brain landmarks on neuroimaging scans. We show that these landmarks can be placed within 2 millimeters on both clinical- and research-grade neuroimaging. We released these coordinates alongside brain images spanning a wide range of participants, scanners, and clinical sites, resulting in a public dataset of over 6,000 landmarks. This protocol enables millimeter quality control of neuroimaging workflows by identifying errors introduced during image processing. In \textbf{Chapter} \textbf{\ref{chap:AutoAFIDs}}, we developed \texttt{AutoAFIDs}, an open-source AI model that automatically detects these brain landmarks with human-level precision. We applied \texttt{AutoAFIDs} to 2,000 brain scans to chart how brain shape varies across age, sex, and disease—revealing millimeter trends across the lifespan. Just as studying a blueprint helps you navigate a dark room, landmarks generated by \texttt{AutoAFIDs} help surgeons navigate brain anatomy with greater reliability. In \textbf{Chapter} \textbf{\ref{chap:afidspred}}, we extended this framework to predict the location of DBS targets based on the spatial arrangement of visible landmarks. By learning the relationship between landmarks and surgical targets, our models estimated target locations on clinical images with millimeter precision. We validated these predictions using ultra-high-resolution imaging—like switching on a flashlight in the dark room. Together, this work delivers open, generalizable tools for millimeter surgical targeting and brain mapping—helping neurosurgeons navigate the brain with greater precision and confidence.
