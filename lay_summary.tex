\onehalfspacing
Imagine walking into a dark room and trying to plug in your phone charger. Without light, you rely on familiar landmarks—a chair, a wall—to guide you. In neurosurgery, a similar challenge arises during deep brain stimulation (DBS), a treatment that modulates dysfunctional brain circuits via millimeter-wide electrodes, functioning like a pacemaker for the brain. DBS targets are small and often difficult to visualize on clinical MRI, and a deviation of just 1–2 millimeters can compromise treatment or cause side effects. To guide electrode placement, surgeons depend on visible anatomical landmarks—reference points that help navigate the brain, much like landmarks in the dark. This dissertation combines brain landmarks with artificial intelligence (AI) to support accurate and interpretable neurosurgical planning. In \textbf{Chapter 2}, we developed and validated a protocol that describes the location of 32 landmarks on neuroimaging scans. We show that these landmarks can be placed within an accuracy of 2 millimeters across low and high quality MRI. We released the coordinates of these landmarks alongside the MRI scans of a wide range of patients, MRI scanners, and clinical sites, resulting in a public dataset of over 6,000 points. Our protocol enables quality control of neuroimaging workflows by identifying millimetric errors introduced during image processing. \textbf{Chapter 3}, we introduced AutoAFIDs, an open-source AI model that automatically detects these 32 landmarks with human-level precision. We applied this tool to over 2,000 brain scans to chart how brain shape varies across age, sex, and disease—revealing millimetric trends across the lifespan. Much like studying a room’s blueprints before entering, this framework helps surgeons navigate brain anatomy more reliably. \textbf{Chapter 4}, we extended this approach to predict the coordinates of DBS targets based on the spatial arrangement of visible landmarks. By learning the relationship between landmarks and surgical targets, our models could estimate target locations on standard clinical MRI. To validate accuracy, we used ultra-high-resolution 7-Tesla MRI—like switching on a flashlight in the dark. Collectively, this work delivers open, generalizable tools for surgical targeting, brain mapping, and neuroimaging validation—helping neurosurgeons navigate the brain with greater precision and confidence.

