\onehalfspacing
A heartfelt thank you to Dr. Jonathan Lau and Dr. Ali Khan, whose guidance, generosity, and vision shaped every stage of my PhD. I feel incredibly grateful to have landed in a lab that offered the freedom to explore, compassionate feedback to refine my ideas, and opportunity to grow into an independent scientist. Jon: your support and investment in my growth allowed me to find my own scientific voice. Your meticulous approach to science and deep care for patients set a standard I will forever carry with me into my academic career and medical training. Ali: you showed me that there is always more than one right way to approach a problem. Your passion for open science and openness to let me shape the technical direction of my work made programming a core part of how I think and contribute as a scientist.

I am especially thankful to Dr. Greydon Gilmore and Dr. Mohamad Abbass. Greydon introduced me to the field of neuromodulation seven years ago, setting me on a path that would ultimately shape this dissertation. He taught me the fundamentals of intraoperative neurophysiology, a skillset that kept me grounded in clinical care throughout my PhD. Mohamad’s mentorship, both in the lab and beyond, was instrumental in my growth. He challenged my thinking, encouraged my ideas, and supported me through the ups and downs of research. I am grateful for his guidance and the clinical perspective he brought to the journey.

I would also like to thank Dr. Terry Peters, who has been a mentor since my undergraduate training. His early support, and his efforts to connect me with scientists at the Robarts Research Institute, helped set this journey in motion. I am grateful as well to Dr. Andreas Horn for his feedback on projects related to this dissertation, and for the opportunity to be involved in the \textit{Stimulating Brains} podcast, which kept me connected to the neuromodulation community.

To my labmates: thank you for the camaraderie and the countless hours spent brainstorming and coding. I learned so much from our conversations, from your curiosity, and from the generous way you approached science. I am especially grateful to those who shared and nurtured my passion for the anatomical fiducials project and data science over the years. Thank you for filling the past years with the kind of moments that may not show up in dissertations but stay with you forever.

To my friends: thank you for keeping me grounded and reminding me there is life beyond the thesis. Bahaa: your friendship means more than I can put into words. I am grateful to have you in my corner.

Last but certainly not least, thank you to my family. Mom, Dad, and Adam: your love, patience, and unwavering belief in me made this possible. Tete and Gedo: despite being hundreds of miles away, you were always a call away supporting my every move. You all taught me how to stay curious, how to persevere, and how to hold on to joy through it all.