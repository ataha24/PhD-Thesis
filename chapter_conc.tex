\chapter{Conclusions and Future Directions}
\newpage
\sloppy
\section{Summary of Contributions}
This dissertation presents a coordinate-based framework for evaluating, localizing, and understanding human brain structure, with a particular focus on neurosurgical applications such as deep brain stimulation (DBS). It advances the use of anatomical fiducials (AFIDs) as a biologically meaningful, extensible, and interoperable system for brain morphometry and machine learning-driven analysis.

Chapter 2 validated and released AFID data, a standardized set of 32 AFIDs sampling multiple brain strictures:
\begin{itemize}
    \item \textbf{Release of AFID Data:} Openly released all AFID data and imaging, addressing a need in the literature for curated brain coordinate data to enable various applications in automation and brain mapping.
    \item \textbf{Benchmarking Registration:} Applied AFID data to evaluate commonly used image registration workflows, demonstrating that registration error varies by landmark and that conventional methods can introduce structure-specific distortions.
\end{itemize}

Chapter 3 developed and evaluated a deep learning pipeline (\texttt{AutoAFIDs}) that predicts AFID coordinates on T1-weighted MRI and packaged that in a BIDS-aware pipeline for large-scale neuroimaging analysis:
\begin{itemize}
    \item \textbf{Localization Evaluation:} Demonstrated millimetric prediction accuracy in test subjects, with error patterns aligning with human inter-rater variability.
    \item \textbf{Registration Evaluation:} Demonstrated millimetric prediction accuracy in test subjects, with error patterns aligning with human inter-rater variability.
    \item \textbf{Applications:} Demonstrated the utility of \texttt{AutoAFIDs} in analyzing anatomical variability across a large lifespan neuroimaging dataset, showing sex- and age-dependent trends and disease-specific structural shifts. Applied PCA to reveal population structure in AFID space, underscoring AFIDs’ potential for brain morphometry and neurodegeneration studies.
\end{itemize}

Chapter 4 applied the AFID coordinate data to localize obscure stereotactic neurosurgical targets, particularly the subthalamic nucleus (STN):
\begin{itemize}
    \item \textbf{Localization Error Analysis:} Quantified localization error of registration-based and CNN-based STN targeting methods using AFID-based ground truth.
    \item \textbf{Dataset-Specific Bias:} Revealed systematic biases in target predictions across imaging modalities (e.g., 7T vs. 1.5T with gadolinium) and subjects.
    \item \textbf{Human vs. Model Comparison:} Compared human annotations to CNN predictions, showing that AFID-based pipelines can approach expert-level accuracy and offer interpretable alternatives to black-box models.
    \item \textbf{Tool Development:} Released a reproducible benchmarking pipeline to support ongoing evaluation of registration and localization tools in neurosurgical contexts.
\end{itemize}
