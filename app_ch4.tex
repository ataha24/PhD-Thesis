\chapter{Chapter 4 Supplementary Content}\label{app:suppcontentch4}
\myappendices{Appendix \ref{app:suppcontentch4}: Chapter 4 Supplementary Content}
\newpage

\section{Quality Control}\label{app:qualitycontrol}
All annotations were quality controlled by AT, JZ, and EC. To facilitate ease of review, a QC interface was created in the form of an HTML file format. This QC file was generated for all the participants in the dataset and contains all annotations (e.g., AFIDs or STN) performed for that subject. We make all QC files availible in: \url{https://github.com/afids/afids-pred/tree/main/data/QC}.

\begin{figure}[hbt!]
    \centering
    \includegraphics[width=0.95\linewidth]{figs/figuresupQC.png}
    \caption{Quality control visualization for anatomical landmark placement. Crosshairs indicate the annotated location of an example fiducial (the infracollicular sulcus [ICS]) in sagittal, coronal, and axial MRI views. Reviewers can toggle between the subject scan and the protocol-defined (template) annotation using the R key while navigating select slices in MRI volume using arrowkeys or slider.}
    \label{fig:figuresupQC}
\end{figure}

\newpage
\section{Machine Learning Model Comparisons}\label{app:qualitycontrol}
We compared two machine learning models for coordinate-to-coordinate regression of the subthalamic nucelus (STN). Ridgre regression exhibited statistically lower errors in Euclidean error when compared to XGBoost. No significant difference was observed in the z-direction. Statistical comparisons were performed using Wilcoxon test (paired samples) with Bonferroni correction (\(\alpha = 0.05/4\))



\begin{figure}[hbt!]
    \centering
    \includegraphics[width=0.95\linewidth]{figs/figuresupridgevsxgboost.png}
    \caption{Comparison of prediction errors between Ridge regression and XGBoost across spatial axes and Euclidean distance (ED). Violin plots show the distribution of per-fold prediction errors for each evaluation metric: mean squared error along the x, y, and z axes (\texttt{mse\_x}, \texttt{mse\_y}, \texttt{mse\_z}), and overall Euclidean distance (ED). Results are grouped by model type. Asterisks indicate the significance of paired comparisons between models using the Wilcoxon signed-rank test (\textit{p}\(<\)0.05: *, \textit{p}\(<\)0.01: **, \textit{p}\(<\)0.001: ***, \textit{p}\(<\)0.0001: ****; ns: not significant).}
    \label{fig:figuresupridgevsxgboost}
\end{figure}